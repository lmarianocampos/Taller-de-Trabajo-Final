% Chapter Template

\chapter{Conclusiones} % Main chapter title

\label{Chapter5} % Change X to a consecutive number; for referencing this chapter elsewhere, use \ref{ChapterX}

En este último capítulo se presentan las conclusiones generales acerca de la respuesta proporcionada por el sistema completo. Además se indica las posibles mejoras y los trabajos pendientes a realizar.
%----------------------------------------------------------------------------------------

%----------------------------------------------------------------------------------------
%	SECTION 1
%----------------------------------------------------------------------------------------

\section{Resultados obtenidos}

%La idea de esta sección es resaltar cuáles son los principales aportes del trabajo realizado y cómo se podría continuar. Debe ser especialmente breve y concisa. Es buena idea usar un listado para enumerar los logros obtenidos.

%Algunas preguntas que pueden servir para completar este capítulo:
%
%\begin{itemize}
%\item ¿Cuál es el grado de cumplimiento de los requerimientos?
%\item ¿Cuán fielmente se puedo seguir la planificación original (cronograma incluido)?
%\item ¿Se manifestó algunos de los riesgos identificados en la planificación? ¿Fue efectivo el plan de mitigación? ¿Se debió aplicar alguna otra acción no contemplada previamente?
%\item Si se debieron hacer modificaciones a lo planificado ¿Cuáles fueron las causas y los efectos?
%\item ¿Qué técnicas resultaron útiles para el desarrollo del proyecto y cuáles no tanto?
%\end{itemize}


%----------------------------------------------------------------------------------------
%	SECTION 2
%----------------------------------------------------------------------------------------
\section{Próximos pasos}

A continuación se detallan los trabajos futuros que se deben tener en cuenta para desarrollar y lograr un sistema global que pueda controlar el caudal en gran cantidad de compuertas que conforman una red de canal, de modo tal que se pueda desempeñar de forma autónoma y eficiente. 
Como se trata de un prototipo, es imprescindible, como primera medida, adaptar este sistema a una compuerta real perteneciente a una red de canal de agua y luego someter a este equipo a pruebas de campo para validar su correcto funcionamiento.

El trabajo a efectuar, posterior a las pruebas de campo, es desarrollar un sistema de software principal con características de escalabilidad, de modo tal que permita añadir o integrar una gran cantidad de “celdas primarias”, ya que la mayoría de las redes de canales de distribución de agua están conformadas por un gran número de compuertas.

Adicionalmente, este sistema deberá poseer la capacidad de gestionar la comunicación entre diferentes celdas primarias, instaladas en cada una de las compuertas de una red, mediante el empleo de la tecnología Lorawan para que el sistema mismo trabaje de forma independiente. También será necesario desarrollar una aplicación móvil que permita gestionar a través del sistema principal la conducción de flujo de agua hacia un determinado canal para abastecerse de dicho recurso y así cumplir con la dotación de riego de canal establecido previamente.


